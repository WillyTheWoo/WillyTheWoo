 \documentclass[11pt]{article}

\usepackage[letterpaper]{geometry} %So margins are not their default huge
\usepackage[parfill]{parskip}    % Activate to begin paragraphs with an empty line
\usepackage{graphicx}
\usepackage{amsmath}
\usepackage{amsfonts}
\usepackage{amssymb}
\usepackage{amsthm}
\usepackage{amsthm} %Allows me to create theorem*-like environments, with no numbering.
\usepackage{commath} %to help me use \eval, for derivatives evaluated at somewhere
\usepackage{ mathrsfs } % to use \mathscr{L} for a Lagrangian
\usepackage{xcolor}
\usepackage{verbatim} %Allows multiline comments with \begin{comment}
\usepackage{tikz}  %for making graphs in LaTeX
\usepackage{enumerate}


% Some custom commands
%
\newcommand{\deldel}[2]{ \frac{ \partial #1}{\partial #2}}  %partial derivative
\newcommand{\dd}[2]{ \frac{d #1}{d #2}}  % derivative
\newcommand{\recip}[1]{\frac{1}{#1}}  %reciprocal
\newcommand{\half}{\recip{2}} %shortcut for 1/2
\newcommand{\ratioone}{\frac{U_{zi}}{U_x}}
\newcommand{\z}{z_1,z_2,..., z_n}
\DeclareMathOperator{\st}{s.t. \,} %shortcut for such that
\DeclareMathOperator{\argmax}{argmax} %argmax
\DeclareMathOperator{\argmin}{argmin} %argmin





 %Fancy-header package to modify header/page numbering 
 %
 \usepackage{fancyhdr}
 \pagestyle{fancy}
 %\addtolength{\headwidth}{\marginparsep} %these change header-rule width
 %\addtolength{\headwidth}{\marginparwidth}
 \lhead{\small\scshape Hieu Nguyen}
 \chead{} 
 \rhead{\footnotesize Family Economics and Macro Behavior Summary} 
 \lfoot{Part \thesection} 
 \cfoot{} 
 \rfoot{\thepage} 
 \renewcommand{\headrulewidth}{.3pt} 
 \renewcommand{\footrulewidth}{.3pt}
 \setlength\voffset{-0.25in}
 \setlength\textheight{648pt}



\begin{document}

\title{Summary for Family Economics and Macro Behavior}
\author{Hieu Nguyen}

%\date{October 5, 2012}


\maketitle
\begin{abstract}
Family economics is a growing and respectable field and have many implications for other parts of economics. 
\end{abstract}



%This line allows the subsections to be delineated by letters in parentheses.  Try commenting it out to see how subsection labels change.
%\def\thesubsection{(\alph{subsection})}

\section{The malthusian and Neoclassical Models}
\indent Mathusian models assumes diminishing returns to increase in the level of population when land and other capital are fixed. His model is consistent with constant returns to the scale of labor and capital, as long as the capital stock does not repond to changes in wages and interest rates. The response of fertility and mortality to changes in income determine the Malthusian supply of population. (Preventive check on population: populations grow more slowly when wages are low because the average person marries later and thereby has fewer children.) The long-run equilibrium wage rate is determined from the population supply curve where the average family has two children. The economy's production function then determines the stationary level of population that is consistent with this long-run wage rate. In this model, tastes for marriages and children determine long-run wages. The long-run wage is stable in the Malthusian model when shocks push the system out of equilibrium (the equilibrium wage is more immune to shocks in the Malthusian system that is the level of population). The Malthusian model does help some in explaining very long-term changes in European wage rates prior to the nineteenth century but fails as fertility fell sharply as wage rates and per capita incomes advanced. 

\indent The Neoclassical growth model incorporates utility maximization and the recognition that changes in the capital stock respond to rats of return on investments (law of motion for capital). But it also assumes that fertility and other dimensions of population growth are independent of wages, incomes, and prices. (a step back from Malthus). The neoclassical equilibrating mechanism works through changes in the rate of investment (compared with changes in the rate of population growth in Malthusian model).

\section{The family and Economic Growth}
 A more relevant growth model is available through combining the best features of the neoclassical and Malthusian models and by adding a focus on investment in knowledge and skills:
 \begin{itemize}
 	\item Parents choose both the number of children and the capital bequeathed to each child: parent's utility depend on their own life-cycle consumption and separately on their degree of altruism per child, the number of children, and the utility of each child. 
	\item Since child rearing is time intensive, the cost of rearing children is positively related to the value of parents' time. Income per capita would rise between the parents' and the child's generations if the total capital bequeathed to each child exceeds the capital inherited by each parent.  
	\item if the number of children is positively related to the income of parents, then this model also has stable steady-state levels of the capital-labor ratio and per capita income, and the steady states depends on variables that change the demand for children. 
	\item He then gives different examples how this model give different implications about Social Security System and tax incidence (see paper for more details). 
	\item Investments in education and other human capital are more productive when past investments are larger. That is, the rates of return on investment in human capital may rise as the stock of human capital grows.
	\item Human capital investment is important since the true ratio of human capita to total capita stock maybe as high as 90 percent or as low as 50 percent. 
 \end{itemize}
 
 \section{Short and Long Cycles}
Marriages, births, and other family behavior respond to fluctuations in aggregate output and prices: children are cheaper during recessions because the value of time spent on children by working mothers is low, investments in education and other human capital are much less pro cyclical(positively correlated with the overall state of the economy) than investments in physical also because the forgone value of time spent in school is cheaper during bad times.  

Family behavior may play more than a negligible role even in generating ordinary business cycles (an increase in the labor supply of married women or young people when household work or school becomes less attractive can induce cyclical responses in aggregate output and other variables). Family behavior is likely to be crucial to long cycles in economic activity: in the modified Malthus-neoclassical model, family choices cause long cycles not only in population growth, but also in capital, output, and other variables if the elasticity of the degree of altruism per child with respect to the number of children declines as families get larger. 

\section{Overlapping Generations}
Throughout history the risks faced by the elderly, young, sick, and unemployed have been met primarily by the family, not by state transfers, private charity, or private insurance. The altruism, love and social norm pressure children, parents, spouses, and other relatives into helping out family members in need. The neglect of childhood and of the intimate relations among parents, children, husbands-wives, and other family members misled these studies sometimes into focusing on minor problems and diverted attention away from some important consequences of the overlapping generations. 



\section{Conclusion}
People spend much of their time in a dependency relation and marriage is a crucial step for most people, children absorb time, energy, and money from their parents, divorce causes economic hardship and mental depression. The message is that family behavior is active, not passive, and endogenous, not exogenous: families have large effects on the economy, and evolution of the economy greatly changes the structure and decisions of families. 


%\begin{thebibliography}{4}
%\end{thebibliography}
 \end{document}

	